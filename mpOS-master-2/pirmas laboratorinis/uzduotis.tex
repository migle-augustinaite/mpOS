\section{Projekto sąlygos}
	Projektuojama interaktyvi OS.
	
	\textbf{Virtualios mašinos} procesoriaus komandos operuoja su duomenimis, esančiais registruose ir ar 		atmintyje. Yra komandos duomenų persiuntimui iš atminties į registrus ir atvirkščiai, aritmetinės (sudėties, atimties, daugybos, dalybos, palyginimo), sąlyginio ir besąlyginio valdymo perdavimo, įvedimo, išvedimo, darbo su failais (atidarymo, skaitymo, rašymo, uždarymo, sunaikinimo) ir programos pabaigos komandos. Registrai yra tokie: komandų skaitiklis, bent du bendrosios paskirties registrai, požymių registras (požymius formuoja aritmetinės, o į juos reaguoja sąlyginio valdymo perdavimo komandos). Atminties dydis yra 16 blokų po 16 žodžių (žodžio ilgį pasirinkite patys).
	
	\textbf{Realios mašinos} procesorius gali dirbti dviem režimais: vartotojo ir supervizoriaus. Virtualios mašinos atmintis atvaizduojama į vartotojo atmintį naudojant puslapių transliaciją. Yra taimeris, kas tam tikrą laiko intervalą generuojantis pertraukimus. Įvedimui naudojama klaviatūra, išvedimui - ekranas. Yra išorinės atminties įrenginys - kietasis diskas.
Vartotojas, dirbantis su sistema, programas paleidžia interaktyviai, surinkdamas atitinkamą komandą. Laikoma, kad vartotojo programos yra realios mašinos kietajame diske, į kurį jos patalpinamos „išorinėmis", modelio, o ne projektuojamos OS, priemonėmis.
	\clearpage