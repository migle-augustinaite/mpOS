\section{Resursai}
	\subsection{Resursų deskriptorius}
		\begin{enumerate}
			\item \textbf{RID} - resurso vidinis vardas.
			\item \textbf{TYPE} - resurso tipas.
			\item \textbf{PROC} - sąrašas procesų, kurie prašė šio resurso.
			\item \textbf{STATE} - resurso būsena.
			\item \textbf{PID} - procesas, sukūręs resursą.
			\textbf{}
		\end{enumerate}
	\subsection{Resursų tipai ir magic number}
		
		(INT) - resursą norima sukurti dėl programinio arba supervizorinio pertraukimo, tačiau ne dėl timer'io pertraukimo.
		(USER) - vartotojiškas procesas.
		
		Resursų tipai, jų aprašymai, susiję procesai ir šių resursų MagicNumber (nurodomas kaip numeris sąraše):\\
		
		\begin{tabular}{| p{1,5cm} |*{3} {p{2,8cm} |} p{5cm}|}
			\hline
			Magic Number	& Pavadinimas	& Laukiantysis Procesas	& Kuriantysis Procesas	& Aprašymas\\
			\hline
			1 				& OSExit		& Start\_Stop			& ProcessKiller			& OS darbo baigimas.\\
			\hline
			2				& ReadBlock		& Get\_Put\_Data		& (bet koks)			& Blokas, esantis atmintyje, įrašomas į duomenų segmentą.\\
			\hline
			3				& WriteBlock	& Get\_Put\_Data		& (bet koks)			& Blokas, esantis duomenų segmente, įrašomas į  vietą, nurodytą atmintyje.\\
			\hline
			4				& ReadWord		& Get\_Put\_Data		& (bet koks)			& Žodis, esantis atmintyje, įrašomas į duomenų segmentą.\\	
			\hline
			5				& WriteWord		& Get\_Put\_Data		& (bet koks)			& Žodis, esantis duomenų segmente, įrašomas į vietą, nurodytą atmintyje.\\
			\hline
			6				& NeedInput		& Chan\_2\_Device		& (INT)					& Įvedimo srautas iš vartotojo.\\
			\hline
			7				& NeedOutput	& Chan\_3\_Device		& (INT)					& Išvedimo srautas į ekraną.\\
			\hline
			8				& ReadFile		& Chan\_1\_Device		& (INT)					& Failo nuskaitymas.\\
			\hline
			9				& OpenFile		& Chan\_1\_Device		& (INT)					& Failo atidarymas.\\
			\hline
			10				& WriteFile		& Chan\_1\_Device		& (INT)					& Rašymas į failą.\\
			\hline
			11				& CloseFile		& Chan\_1\_Device		& (INT)					& Failo uždarymas.\\
			\hline
			12				& DeleteFile	& Chan\_1\_Device		& (INT)					& Failo ištrinimas pagal nurodytą deskriptorių.\\
			\hline
			13				& SeekFile		& Chan\_1\_Device		& (INT)					& Pozicijos faile pakeitimas.\\
			\hline
			14				& ProgramStart	& Loader				& (INT)					& Programos pakrovimas į atmintį.\\
			\hline
			15				& ProgramHalt	& ProcessKiller			& (INT)					& Programos sustabdymas.\\
			\hline
			
		\end{tabular}
		
		\begin{enumerate}
			\item \textbf{OSExit} - OS darbo baigimas. Laukia Start\_Stop procesas, kuria - ProcessKiller. \\
			
			\textbf{Resursai, skirti darbui su vidine atmintimi}: Kuriantys procesai gali būti bet kokie, o laukiantis - visada Get\_Put\_Data.
			
			\item \textbf{ReadBlock} - Blokas, esantis atmintyje, atmintyje, įrašomas į duomenų segmentą.
			\item \textbf{WriteBlock} - Blokas, esantis duomenų segmente, įrašomas į vietą, nurodytą atmintyje.
			\item \textbf{ReadWord} - Žodis, esantis atmintyje, įrašomas į duomenų segmentą.
			\item \textbf{WriteWord} - Žodis, esantis duomenų segmente, įrašomas į vietą, nurodytą atmintyje.\\
			
			\textbf{Resursai, skriti darbui su srautais}:
			
			\item \textbf{NeedInput} - Įvedimo srautas iš vartotojo. Laukia Chan\_2\_Device, kuria - (INT). 
			\item \textbf{NeedOutput} -Išvedimo srautas į ekraną.
			Laukia Chan\_3\_Device, kuria - (INT).\\
			
			\textbf{Resursai, skirti darbui su failais}: Kuriantysis procesas visada (INT), o laukiantysis - Chan\_1\_Device.
			
			\item \textbf{ReadFile} - Failo nuskaitymas.
			\item \textbf{OpenFile} - Failo atidarymas.
			\item \textbf{WriteFile} - Rašymas į failą.
			\item \textbf{CloseFile} - Failo uždarymas
			\item \textbf{DeleteFile} - Failo ištrinimas pagal nurodytą deskriptorių.
			\item \textbf{SeekFile} - Pozicijos faile pakeitimas.\\
			
			\textbf{Resursui, skirti darbui su programomis}:
			
			\item \textbf{ProgramStart} - Programos pakrovimas į atmintį. Laukia Loader procesas, kuria - (INT).
			\item \textbf{ProgramHalt} - Programos sustabdymas. Laukia ProcessKiller procesas, kuria - (INT).
			
		\end{enumerate}
		
		
	\subsection{Resursų paskirstytojas}
		Resursų paskirstytojas suteikia paprašytus resursus procesams pagal prioritetus, jo skirstymo pabaigoje kviečiamas procesų paskirstytojas. 
	\subsection{Resursų primityvai}
		\begin{enumerate}
			\item \textbf{Kurti resursą} - procesas kuria resursą. Perduodami parametrai yra tokie: nuoroda į proceso tėvą, resurso vidinis vardas. Resursas pridedamas prie bendrojo resursų sąrašo, taip pat prie tėvo sukurtų resursų sąrašo, sukuriamas resurso elementų sąrašas ir kuriamas laukiančių procesų sąrašas.
			\item \textbf{Prašyti resurso} - procesui paprašius resurso, jis užsiblokuoja ir yra įtraukiamas į laukiančiųjų resurso procesų sąrašą.
			\item \textbf{Atlaisvinti resursą} - primityvą kviečia procesas, kuris nori nereikalingą resursą arba perduotį informaciją kitam procesui. Primityvo pabaigoje kviečiamas resursų paskirstytojas.
			\item \textbf{Naikinti resursą} - resurso deskriptorius išmetamas iš tėvo sukurtų resursų sąrašo, bendrojo resursų sąrašo, atblokuojami procesai, kurie laukė šio resurso, sunaikinamas pats deskriptorius.
		\end{enumerate}
		